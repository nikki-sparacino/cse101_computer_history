\documentclass[letterpaper, 10 pt, conference]{IEEEconf}
\title{\LARGE \bf
COMPUTER HISTORY\\
\large
}

\author{Group Number 2\\
\small Nicholas Medina\\
\small Nikki Sparacino\\
\small Madisen Good\\
\small Brandon Hunt\\
}

\begin{document}

\maketitle 

\section{Abstract}

The history of gaming, specifically computer gaming, is very young when compared
to the rest of human history, yet its progress has been exponential since the first
game made. The hardware, and software of each generation of gaming have very deep
backgrounds and are the foundation for the current generation.

\section{Introduction}

Gaming is now one of the biggest industries in the world making \$116 billion dollars
in 2017 and is expected to keep growing. As big as it is now the foundation of how gaming
became to be is not widely know. This documentaims to be a guide on how gaming developed from
tic-tac-toe to 100+ gigabyte games by looking at the history of the hardware, software and 
time periods of computer games.

\section{Time period}
Computer games have been around as long as computers have been invented. Games were made to test certain programs and for demonstations. In the 1960's, computer scientists designed simple games mainly for research. The proffessors at MIT played games like 3D tic-tac-toe and Moon Landing. Games were mainly played on the IBM 1560. Moves for the game were made by punch cards. Computer games did not reach much popularity until the 1970's-1980's. This was started when Rhalph Baer invented the "Brown Box". The Brown Box was used by connecting to a television to play the game. The 1970's was also known the era of mainframe computer games. In 1971 Steve Russell's "SpaceWar!" led to the creation of "Computer Space". Computer Space used no microprocessor, RAM, or ROM. Since the 1980's video games have been considered one of the main forms of entertainment. This is when home video game consoles started to gain popularity. The Atari and Nintendo systems were created in the 1980's and are still very popular today. In the 1990's is when handheld gaming devices gained popularity. The Nintendo Game Boy was invented in 1990 and is one of the longest-lived video game consoles. Between 1987 and 999 most gaming consoles were 16-bit models. From 1993 to 2006 gaming consoles changes to 32 and 64-bit. During this time is when gaming on mobile phones started to develop. Between 1998 and and 2013 is when online gameing and mobile games became a major part of pop culture. Mobile games and gaming consoles are used by a large portion of the population now. From 2013 to now is when major consoles like the Wii U, Xbox One, and Playstation 4 became some of the most popular ways for people to play video games. PC gaming still holds a large majority of games but mobile games and consoles are easily available for the majority of the population.
\section{Computer Hardware}
In 1972, Atari released Pong,  the first commercially successful video game. 
Based off of the 1958 Tennis for Two game, which ran on an analog computer, 
Pong initially was built  with a Hitachi black-and-white television with wires
soldered to make the necessary circuit. This version did not require a computer
, microprocessor, or any software. Atari later refined this and made the home 
version of Pong using custom circuits.

Atari later built the Atari 2600 prototype in 1977, using a system containing
the MOS Technology 6502 CPU, an 8-bit microprocessor. It wasn’t the first to 
use a microprocessor and removable game cartridges, but helped establish the 
standard.

The first successful hand-held console was the Game Boy, created by Nintendo. 
It’s CPU was a custom 8-bit microprocessor with a clock frequency of 4.19 MHz.
It contained 8KiB RAM and had a Reflective LCD display with a resolution of 
160 × 144 pixels. It was the first henad-held console to use a game-link cable.

Comparing the Game Boy to the Nintendo Switch, released in 2017, shows how much
the hardware for games has improved in the last 30 years. The CPU for the Switch 
has 4 CPU-cores with 32 KiB data (2-way set-associative) and 48 KiB instruction 
per core, a big improvement from an 8-bit microprocessor. Going from an LCD display
with a resolution of 160 × 144 to a multi-touch capacitive touch LCD screen with
a resolution of 1280 x 720 is a huge upgrade, producing realistic and detailed graphics
that weren’t thought possible when the Game Boy was made.
\section{Computer Software}
One of the first video games created was OXO, or tic-tac-toe and was created by Alexander 
Douglas in 195e. The game was created on the Electronic Delay Storage Automatic Calculator 
(EDSAC) which utilized sheets of paper with holes punched into it known as punch tape. The 
computer would read these punch sheets then project them through it’s cathode ray tube onto 
a screen for the player to read.

Jump ahead to the 90’s, with one of the biggest games Doom. This time period is when the use of 
game engines became popular. Game engines are software developed environments to specifically used 
to create games. Doom was created using its own engine titled the Doom engine. The engine was developed 
in 1993 and was written in the C language. 

In modern days we now use consoles or applications for computers like STEAM to play and download games. 
For the current generation of gaming one of the most popular game engines utilized is AnvilNext. This 
engine is used in Rainbow Six: Siege, one of the biggest games available to play in the context of the 
documents date
\section{Conclusion}

From tic-tac-toe to Rainbow Six: Siege, gaming has gone a long way in a short amount of time.
With revolutions in hardware that allowed for games to become bigger and more graphically intesive,
to the use of game engines to help make building working games easier for developement. Thanks to the 
proffessorsat MIT gaming is now more than just a demonstration but peoples income or hobby but that it
can become much more.

\end{document}
