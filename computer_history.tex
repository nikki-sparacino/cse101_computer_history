\documentclass[letterpaper, 10 pt, conference]{IEEEconf}
\title{\LARGE \bf
COMPUTER HISTORY\\
\large The history of computer gaming..
}

\author{Group Number 2\\
\small Nicholas Medina\\
\small Nikki Sparacino\\
\small Madisen Good\\
\small Brandon Hunt\\
}

\begin{document}

\maketitle 

\section{Introduction}
Nick
\section{Time period}
Maddie
\section{Computer Hardware}
In 1972, Atari released Pong,  the first commercially successful video game. Based off of the 1958 Tennis for Two game, which ran on an analog computer, Pong initially was built  with a Hitachi black-and-white television with wires soldered to make the necessary circuit. This version did not require a computer, microprocessor, or any software. Atari later refined this and made the home version of Pong using custom circuits.

Atari later built the Atari 2600 prototype in 1977, using a system containing the MOS Technology 6502 CPU, an 8-bit microprocessor. It wasn’t the first to use a microprocessor and removable game cartridges, but helped establish the standard.

The first successful hand-held console was the Game Boy, created by Nintendo. It’s CPU was a custom 8-bit microprocessor with a clock frequency of 4.19 MHz. It contained 8KiB RAM and had a Reflective LCD display with a resolution of 160 × 144 pixels. It was the first henad-held console to use a game-link cable.

Comparing the Game Boy to the Nintendo Switch, released in 2017, shows how much the hardware for games has improved in the last 30 years. The CPU for the Switch has 4 CPU-cores with 32 KiB data (2-way set-associative) and 48 KiB instruction per core, a big improvement from an 8-bit microprocessor. Going from an LCD display with a resolution of 160 × 144 to a multi-touch capacitive touch LCD screen with a resolution of 1280 x 720 is a huge upgrade, producing realistic and detailed graphics that weren’t thought possible when the Game Boy was made.
\section{Computer Software}
Brandon
\section{Conclusion}
everyone
\section{References}

\end{document}
